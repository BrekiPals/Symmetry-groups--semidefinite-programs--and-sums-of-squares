\documentclass[]{article}
\usepackage{MT}

\addbibresource{references}
\begin{document}

\title{Symmetry groups, semidefinite programs, and sums of squares}
\author{Breki Pálsson and Simon Wegan}
\date{20. April 2023}

\maketitle

\section{Introduction}
We will look at a fundamental problem in real algebraic geometry i.e. the existence and computation of a
representation of a multivariate polynomial as a sum of squares (SOS). In other words, the
question of finding $p_i \in \R[x], i = 1, \cdots, N$ such that

\[f(x) = \sum_{i=1}^{N}(p_i(x))^2.\]

This problem has applications in many fields of applied mathematics, such as continuous and combinatorial optimization as well as being theoretically interesting. 


We will show a method that exploits symmetries in polynomials and semidefinite programming (SDP) in order to get a reduction in the problem size. 
Two reasons are for this, firstly we get a faster solution since the time complexity of SDF has been shown to be polynomial \todo{do we have to cite this?}, 
secondly smaller problem size give more accurate solutions may that be do to numerical conditioning or numerical errors.


The paper outlines the theoretical background and gives the reader examples explaining the definitions step by step in order to present an algorithm that is able to use the 
symmetric properties of a polynomial that is invariant with respect to a certain representation and produces a solution to a semidefinite program given certain constraints.

\subsection*{The problem}

Given a polynomial that has symmetries we want to use semidefinite programming in order to find

We will use representation theory to define a good notion of symmetries. In particular we will look at representations
$\sigma:G\rightarrow Aut(S_+^N)$ that preserve $\mathcal{S}_+^N$ and are \textit{induced} by a representation $\rho:G\rightarrow GL(\R^N)$, 
that is 
\[\sigma(g)(\mathcal{S}_+^N)\subseteq \mathcal{S}_+^N, \forall g\in G , \,\,\, \sigma(g)(X) := \rho(g)^TX\rho(g) \,\,\, X\in \mathcal{S} , g\in G.\]

These representation will be of this kind for most practical instances. The paper shows that with a convenient change of coordinates every 
invariant matrix will be block diagonal enabling us to go from one big problem to a couple of smaller ones. 
\begin{definition}
    Given a finite group $G$, and an associated linear representation $\sigma:G\rightarrow Aut(\mathcal{S}^N)$, a semidefinite
    optimization problem of the form $F:=\min_{\mathcal{L}\cap \mathcal{S}_+^N}\left<C,X\right>.$ is called invariant with 
    respect to $\sigma$, if the set of feasible matrices $\mathcal{L}\cap \mathcal{S}_+^N$ and the cost function $\left<C,X\right>$are invariant with respect to $\sigma$
\end{definition}
\begin{definition}
    We define the fixed-point subspace of $\mathcal{S}^N$ as the subspace of all invariant matrices, 
    \[\mathcal{F} := \{X\in\mathcal{S}^N | X= \sigma(g)(X)\,\,\, \forall g \in G\}\] 
    and the associated semidefinite program 
    \[F_\sigma := \min_{X\in\mathcal{F}\cup\mathcal{L}\cap \mathcal{S}_+^N } \left<C,X\right>\]
\end{definition}

The paper proves that the solution a SDP which is invariant with respect to a linear representation $\sigma$ of the kind we have discussed and it's sigma SDP outlined above give the same solution. 
Thus we can restrict our set of feasible matrices from $\mathcal{L}\cap \mathcal{S}_+^N$ to $ X\in\mathcal{F}\cup\mathcal{L}\cap \mathcal{S}_+^N$ making the problem simpler.

Furthermore every linear representation of a finite group $G$ has a canonical decomposition as a direct sum of irreducible representations, i.e.
\[\rho = m_1\vartheta_i\oplus m_2\vartheta_2 \oplus \cdots \oplus m_h\vartheta_h.\]
The paper showed that with these constraints we can actually find a change of coordinates so that all the matrices 
in the SDP have block diagonal form, the problem therefore collapses into a collection of smaller 
optimization problems, which are much easier to solve as outlined below 
\[F = \min_{X\in\mathcal{L}, X_i \in \mathcal{S}_+^{m_i} }\sum_{i=1}^{h} n_i\left<C_i,X_i\right>.\]

Next we look at some invariant theory in order to simplify the problem even further. 

Assume we are interested in finding the sum of squares decomposition of a polynomial $f(\textbf{x})$ of degree $2d$ in $n$
variables which is invariant with respect to a linear representation $\vartheta:G\rightarrow GL(\R^n)$, i.e. $f(\textbf{x}) = f(\vartheta(g)\textbf{x}), \forall g\in G$.
The set of all such invariant polynomials is the invariant ring, denoted by $\R[\textbf{x}]^G$. We could jump to the conclusion that if a $f\in \R[\textbf{x}]^G$ can be written as a 
sum of squares $f(\textbf{x}) =\sum_{i=1}^{h}(p_i(\textbf{x}))^2$ then the polynomials $p_i$ must also be in $\R[\mathbb{x}]^G$. Unfortunately this turns out not to be true. We can however
use a technique that allows us to couple together squares so they decompose into invariant components. This means we can write 
\[f(\textbf{x}) = \sum_{i=1}^{h}\left<Q_i,P_i(\mathbf{x})\right>,\]
with $P_i\in (\R[\mathbf{x}]^G)^{m_i\times m_i}$
where the $Q_i$ are as always positive semidefinite.


Furthermore using the Hironaka decomposision of the invariant ring, i.e.
\[\R[\mathbf{x}]^G = \bigoplus_{j=1}^{t}\eta_j(\textbf{x})\R[\theta_1(\textbf{x}), \cdots, \theta_n(\mathbf{x})]\] 
where $\theta_i(\mathbf{x})$, $\eta_i(\mathbf{x})$ are called primary and secondary invariants, respectively, 
we can actually represent $f(\mathbf{x})$ uniquely as a function with the invariants as inputs. i.e.

\[\tilde{f}(\theta,\eta) = \sum_{i=1}^{h}\left<Q_i,\tilde{P}_i(\theta,\eta)\right>, \tilde{P_i} \in (\tilde{T})^{m_i\times m_i}, \tilde{T} = \sum_{i=1}^{h}\eta_j\R[\theta]\],
where the $Q_i$ are as always positive semidefinite.
In practice very often the group representations are so-called reflection groups. In this situation the invariant ring is 
isomorphic to a polynomial ring and the secondary invariants are not needed.

The last 



\section{Algorithm}
Here we will present an algorithm that is the result of the paper and later explain certain concepts that we need to define in order to understand the algorithm

\subsection*{Algorithm I}


\textbf{Input:} Linear representation $\vartheta$ of a finite group $G$ on $\R^n$.

\begin{enumerate}
    \item Determine all real irreducible representations of $G$.
    \item Compute primary and secondary invariants $\theta_i,\eta_j$. 
    \item For each non-trivial irreducible representation compute the basis $b_1^i,\cdots, b_{r_i}^i$ of the module of equivariants.
    \item For each irreducible representation $i$ compute the corresponding matrix $\Pi_i$.
\end{enumerate}
\noindent
\textbf{Output:} Primary and secondary invariants $\theta,\eta$ and the matrices $\prod_i$.

\subsection*{Algorithm II}

\textbf{Input: } Primary and secondary invariants $\theta,\eta$, matrices $\prod_i$ and $f \in \R[\theta]^G$.

\begin{enumerate}
    \item Rewrite $f$ in fundamental invariants giving $\tilde{f}(\theta,\eta)$.
    \item For each irreducible representation determine $w_i(\theta)$ and thus the structure of the matrices $S_i\in \R[\theta]$.
    \item Find a feasible solution of the semidefinite program corresponding to the constraints.
\end{enumerate}
\noindent
\textbf{Output:} SOS matrices $S_i$ providing a generalized sum of squares decomposition of $\tilde{f}$.



Algorithm I does the preprocessing for our problem while the second lgor, only having a linear representation as an input
this means that we can run algorithm I one time and then 


une the outputs for finding solution for 
all polynomials that are invariant with respect to this particular representation.


\section{Example}
We will demonstrate the efficacy of this algorithm with an example.
\section{Conclusion}
Although it might seem cumbersome to find all the invariants of a representation.... 



\[ \min_{X\in\mathcal{L}\cap \mathcal{S}_+^N } \left<C,X\right> \rightarrow 
\min_{X\in\mathcal{F}\cap\mathcal{L}\cap \mathcal{S}_+^N } \left<C,X\right> \rightarrow
\min_{X\in\mathcal{L}, X_i \in \mathcal{S}_+^{m_i} }\sum_{i=1}^{h} n_i\left<C_i,X_i\right>\]
\[f(\mathbf{x}) = \sum_{i= 1}^{r} (p_i(\mathbf{x}))^2 
                = \sum_{i=1}^{h}\left<Q_i,P_i(\mathbf{x})\right> 
                = \sum_{i=1}^{h}\left<Q_i,\tilde{P_i}(\mathbf{x})\right>
                = \sum_{i=1}^{h}\left<S_i(\theta),\Pi_i(\theta,\eta)\right> 
                \rightarrow f_j(\theta) = \sum_{i=1}^{h}\left<S_i(\theta),\Pi_i^j(\theta)\right>  \]

\begin{definition}
    Let $S\in \R[\textbf{x}]^{m\times m}$ be a symmetric matrix, and $\textbf{y} = [y_1,\cdots,y_m]$ be new indeterminants. The matrix $S$ is a sum of squares (SOS) matrix if the scalar polynomial $\textbf{y}^TS\textbf{y}$ 
    is a sum of squares in $\R[\textbf{x},\textbf{y}]$.   
\end{definition}

\begin{definition}
    
\end{definition}

\cite{Gatermann_2004}



Hello World


\end{document}